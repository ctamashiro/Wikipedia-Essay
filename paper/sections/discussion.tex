
\subsection{Interpretation of Results}

Our results confirm that multilingual Wikipedia summaries reflect culturally and linguistically distinct framing rather than direct translation. English summaries tended to provide broader contextual detail, Spanish summaries often highlighted institutional or political elements, and Japanese summaries favored concise structural information. These differences show that historical narratives are constructed differently across language editions, shaped by editorial norms and cultural communication styles.

\subsection{Implications}

These findings have implications for how global audiences consume historical information. Readers relying on different language editions may receive substantively different interpretations of major historical events. For Wikipedia, this underscores the need for improved cross-language alignment tools and awareness among editors. For researchers, the results demonstrate the value of proxy-based multilingual comparison for detecting subtle forms of linguistic bias.

\subsection{Limitations}

This study is limited by its use of short 500-character summaries, which provide only a narrow window into full article content. The number of topics compared was also small, and no translation normalization or automated linguistic analysis was used. Future work could expand the dataset, incorporate NLP methods, or compare full article structures to achieve a deeper understanding of cross-lingual historical framing.
