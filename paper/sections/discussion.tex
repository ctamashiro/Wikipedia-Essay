% Discussion
% Interpret your results, discuss implications, acknowledge limitations

\label{sec:discussion}

\subsection{Interpretation of Results}

The results of our analysis show that Wikipedia’s multilingual articles are not simple translations of one another but reflect distinct linguistic and cultural framing. Across English, Japanese, and Spanish, the same topics differed in tone, detail, and emphasis—particularly in politically or historically sensitive subjects. These differences suggest that editorial decisions, community norms, and cultural context shape how knowledge is represented. Our findings answer the main research question by confirming that language bias significantly influences how information is structured and presented across languages, even within a supposedly neutral platform like Wikipedia.

\subsection{Implications}

These findings have important implications for how readers interpret global information sources. For Wikipedia, understanding the extent of language bias highlights the need for greater cross-lingual collaboration among editors and tools that flag inconsistencies across versions. For researchers, this study underscores the value of examining multilingual content to uncover systemic bias. In practice, our results demonstrate the importance of critical digital literacy—users should recognize that the “neutral” presentation of facts may differ based on the language they read.

\subsection{Limitations}

Our analysis was limited in both scale and automation. The Wikipedia API program retrieved only short 500-character summaries, which capture limited context. The study also compared a small number of languages, meaning our findings cannot be generalized across all of Wikipedia.
