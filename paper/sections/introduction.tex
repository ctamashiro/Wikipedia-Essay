
Wikipedia is one of the most widely used information platforms in the world, available in over 300 languages and read by millions of users daily. Despite its mission of neutrality and open collaboration, numerous studies and examples have shown that the way information is written and presented can differ depending on the language edition. These differences raise questions about how cultural and linguistic contexts influence the perceived objectivity of shared knowledge.

This paper focuses on multilingual differences within Wikipedia, particularly in the domain of historical topics. While Wikipedia strives for a neutral point of view, articles about historical events and periods often vary in tone, emphasis, and content depth across different languages. These inconsistencies may influence how readers understand major historical topics, shaping perceptions in ways that reflect broader cultural or editorial priorities.

Understanding these linguistic differences is important because Wikipedia serves as a foundational information source for users around the world. When articles differ across language editions, readers may receive culturally filtered or uneven perspectives on shared global history. A systematic examination of these differences helps reveal how language influences access to balanced and accurate information.

This study distinguishes between the broader motivating research question and the specific operationalized question that can be directly measured with our dataset:

\textbf{Motivating Research Question:}  
\emph{How do different language editions of Wikipedia shape the way historical topics are presented across cultures?}

\textbf{Operationalized Research Question:}  
\emph{How do word count, word choice, and content emphasis differ in History-related Wikipedia summaries across English, Japanese, and Spanish, as retrieved through the Python Wikipedia API?}

The main contributions of this work are as follows:
\begin{itemize}
    \item A multilingual comparison of History-related Wikipedia summaries in English, Japanese, and Spanish.
    \item Empirical evidence showing how linguistic and cultural factors influence tone, detail, and emphasis across language editions.
    \item A proxy-based framework for future computational measurement of multilingual Wikipedia bias.
\end{itemize}

The rest of this paper is organized as follows. 
Section~\ref{sec:related} reviews related work on multilingual Wikipedia bias and governance. 
Section~\ref{sec:methodology} describes our data collection process and proxy-based comparison approach. 
Section~\ref{sec:results} presents our empirical findings with respect to our operationalized research question. 
Section~\ref{sec:discussion} discusses broader implications, limitations, and interpretive considerations. 
Finally, Section~\ref{sec:conclusion} summarizes our conclusions and notes directions for future work.
