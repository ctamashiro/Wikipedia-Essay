% Introduction
% Structure: Problem/Motivation -> Background -> Research Questions -> Contributions -> Paper Outline

\label{sec:introduction}

% Motivating example or opening
Wikipedia is one of the most widely used information platforms in the world, available in over 300 languages and read by millions of users daily. Despite its mission of neutrality and open collaboration, numerous studies and examples have shown that the way information is written and presented can differ depending on the language edition. These differences raise questions about how cultural and linguistic contexts influence the perceived objectivity of shared knowledge.

% The problem/question
This paper investigates the presence of language bias within Wikipedia articles across different languages. Although Wikipedia strives to maintain a neutral point of view, the content, tone, and emphasis of articles can vary significantly depending on the language of publication. Such inconsistencies may alter how users understand global topics, particularly in fields like politics, history, and science.

% Why it matters
Understanding language bias in multilingual platforms is critical because it affects access to accurate and balanced information worldwide. When information differs across language editions, readers may receive culturally filtered or incomplete perspectives. Addressing these discrepancies can promote greater transparency, equity, and inclusivity in digital knowledge sharing.

% Research questions
This paper investigates the following research questions:
\begin{enumerate}
    \item What Wikipedia topics show the most variation in language (e.g., politics, history, science)?
    \item How does language bias affect access to reliable information across underrepresented languages?
    \item How do editors’ demographics and linguistic backgrounds shape the content and tone of Wikipedia articles?
\end{enumerate}

% Contributions
The main contributions of this work are:
\begin{itemize}
    \item A comparative analysis of Wikipedia articles across three language editions (English, Japanese, and Spanish) using the Wikipedia API.
    \item Evidence demonstrating how linguistic and cultural differences influence tone and emphasis in multilingual content.
    \item A framework for future computational analysis of bias using natural language processing tools.
\end{itemize}

% Paper outline
The rest of this paper is organized as follows. 
Section~\ref{sec:related} reviews related research on Wikipedia bias and multilingual governance. 
Section~\ref{sec:methodology} describes our data collection and comparison approach using the Wikipedia API. 
Section~\ref{sec:results} presents our main findings. 
Section~\ref{sec:discussion} discusses the broader implications and limitations of our study. 
Finally, Section~\ref{sec:conclusion} concludes and outlines directions for future work.
