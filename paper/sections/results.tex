
This section presents findings addressing our operationalized research question: \emph{How do word count, word choice, and topical emphasis differ in History-related Wikipedia summaries across English, Japanese, and Spanish?} All results derive from the API-collected summaries described in Section~\ref{sec:methodology}.

\subsection{Word Count Differences}

English summaries were consistently the longest across all topics. For example, the English summary for “World War II” contained 498 characters, compared to 452 in Spanish and 311 in Japanese. A similar pattern appeared for all historical topics tested. This suggests that English Wikipedia tends to provide more context within the summary, while Japanese summaries are more concise and Spanish summaries fall in between.

\subsection{Word Choice Differences}

Distinct linguistic patterns emerged across languages.  
English summaries frequently used globally standardized historical vocabulary such as “allies,” “conflict,” and “industrialization.”  
Spanish summaries used terms emphasizing political actors and institutions (e.g., “potencias,” “gobiernos”).  
Japanese summaries employed neutral, descriptive phrasing and avoided evaluative vocabulary.

These differences reflect each language community’s stylistic norms and conventions in presenting historical information.

\subsection{Differences in Emphasis and Focus}

English summaries tended to foreground causes, consequences, and chronological structure.  
Spanish summaries sometimes emphasized political interpretation or regional context.  
Japanese summaries often emphasized definitions, periods, or structural overviews rather than political framing.

These patterns suggest that different language editions prioritize different narrative elements, even for widely shared historical subjects.

\subsection{Summary of Findings}

Across all three proxies—word count, word choice, and emphasis—multilingual summaries showed systematic differences. English provided the most detail, Spanish offered interpretive nuance, and Japanese prioritized concise and neutral description. These findings demonstrate that even short summaries exhibit culturally and linguistically shaped variation.
