% Conclusion
% Briefly summarize: problem, approach, findings, contributions
% End with a forward-looking statement

\label{sec:conclusion}

This research investigated how language bias influences the content and interpretation of Wikipedia articles across different language editions. Although Wikipedia is built on the principle of neutrality, our analysis revealed noticeable differences in emphasis, tone, and detail between English, Japanese, and Spanish versions of the same topics. Using the Wikipedia API, we collected and compared summaries to illustrate how cultural and linguistic contexts shape what is presented as “neutral” knowledge. 

Our findings demonstrate that Wikipedia is not simply translated content but a reflection of localized editorial norms and community perspectives. This highlights the importance of considering linguistic diversity when evaluating information reliability and accessibility. By showcasing how even factual entries can differ by language, this project contributes to the broader understanding of systemic and cultural bias in online information platforms. 

Future work could expand this approach by integrating natural language processing tools to quantify bias across a larger dataset and by exploring strategies for promoting greater consistency and inclusivity in multilingual digital knowledge systems.
