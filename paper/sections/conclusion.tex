
This study examined how Wikipedia’s multilingual editions present historical topics differently by analyzing 500-character summaries in English, Japanese, and Spanish. Using word count, word choice, and topical emphasis as measurable proxies, we identified systematic differences across languages. English summaries provided the greatest level of detail, Spanish summaries offered more interpretive framing, and Japanese summaries emphasized concise and neutral description.

These findings show that even short summaries reflect culturally shaped editorial choices. Wikipedia, despite its neutrality policy, presents distinct linguistic versions of history depending on the language edition. Our work provides a reproducible framework for detecting such differences using simple API-based methods.

Future research should expand the scale of analysis, incorporate automated translation alignment or NLP-based metrics, and explore full-article structural differences to better understand multilingual knowledge representation.
