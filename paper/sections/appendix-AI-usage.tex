% AI Usage Documentation
% Document how you used AI tools in your research and writing process
% Be transparent and specific

\subsection{Literature Review}

We used an AI-assisted literature review workflow (see the file \href{../literature/literature-review.prompt.md}{literature-review.prompt.md}) to summarize and analyze key sources about bias and multilingual representation on Wikipedia. The agent processed three core articles from Wikipedia, Johns Hopkins University, and New Scientist. It extracted summaries, methodologies, and key findings, which were then manually reviewed and incorporated into our related work section. The AI also helped generate BibTeX entries for our references file.

\subsection{Data Analysis}

AI assistance was not directly used to analyze the data. Instead, we manually compared outputs from the Wikipedia API in English, Japanese, and Spanish to identify qualitative differences in phrasing, tone, and content. However, we plan to extend this work by integrating a natural language processing (NLP) model in the future to automatically detect patterns of bias.

\subsection{Writing Assistance}

We used ChatGPT to support various parts of the writing process, including improving the clarity and structure of the abstract, generating section outlines for consistency with the provided ACM LaTeX template, and editing sentences for readability. All content was reviewed and revised by group members before inclusion in the final paper.

\subsection{Code Development}

ChatGPT was used for limited code guidance when setting up and debugging the Wikipedia API script (week7.py). The AI provided suggestions on how to retrieve summaries, handle different languages, and format results in a readable output. The final implementation was reviewed and tested by our group.

\subsection{Verification}

All AI-generated material was reviewed and verified by the authors. Summaries were cross-checked with original sources, and all code suggestions were tested for accuracy. No AI-generated text or code was used without human editing or confirmation. Factual claims were verified against primary data and research papers to ensure accuracy and integrity.
