
Research on Wikipedia has examined its community structure, governance mechanisms, and the systemic biases that emerge across topics and languages. Scholars widely acknowledge that while Wikipedia aspires to neutrality, the platform reflects the cultural, linguistic, and ideological perspectives of its contributors. Recent work in particular highlights how multilingual editions of Wikipedia diverge due to translation asymmetries, local editorial norms, and culturally specific framing choices.

\subsection{Wikipedia Governance and Community Culture}

Reagle’s foundational study, \textit{Good Faith Collaboration}~\cite{reagle2010good}, examines how Wikipedia’s norms, policies, and consensus-based editing processes shape participation. While these structures promote openness, they also produce uneven influence, often privileging dominant language communities and long-term editors. Prior work on governance demonstrates that cultural preferences embedded in community norms can subtly affect how content is written, updated, and moderated.

\subsection{Linguistic and Ideological Bias on Wikipedia}

Beyond governance, recent analyses directly examine how linguistic and ideological biases manifest in Wikipedia articles. The Wikipedia article on ideological bias~\cite{wikipedia2024ideologicalbias} outlines patterns of political framing across language editions, while empirical studies such as the Johns Hopkins University Hub report~\cite{jhu2025wikipediabias} show that multilingual articles differ not only in length but also in emphasis and tone. New Scientist~\cite{newscientist2016wikipedialanguage} similarly reports that language editions selectively add or omit facts, creating culturally inflected versions of the same topic. Collectively, this work shows that multilingual Wikipedia cannot be treated as a uniform body of information.

\subsection{Our Work in Context}

While existing research identifies systemic and linguistic bias broadly, few studies have operationalized multilingual comparison through short-form summaries obtained directly through the Wikipedia API. Our project addresses this gap by examining English, Japanese, and Spanish summaries of historical topics using measurable proxies—word count, word choice, and topical emphasis. By focusing on concise API-retrieved summaries, our work offers a reproducible, fine-grained perspective on how historical information diverges across languages. This connects high-level scholarship on Wikipedia bias with a concrete, computational methodolog
