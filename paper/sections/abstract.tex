Wikipedia offers global access to knowledge across more than 300 languages, but substantial evidence shows that articles differ in tone, emphasis, and detail across language editions. This study investigates how historical topics are represented differently across English, Japanese, and Spanish Wikipedia by analyzing short summaries retrieved through the Wikipedia API. We operationalize multilingual variation through three measurable proxies: word count, word choice, and topical emphasis. Our findings show that English summaries provide the most contextual detail, Spanish summaries often emphasize political or institutional actors, and Japanese summaries favor concise and neutral phrasing. These consistent patterns demonstrate that linguistic and cultural factors shape how historical information is framed across languages. The study contributes a reproducible, proxy-based methodology for detecting multilingual variation and highlights the need for cross-lingual coordination to promote equitable global knowledge representation.
