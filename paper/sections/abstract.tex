% Abstract: 150–250 words summarizing your entire paper
% Include: problem, methods, key findings, contribution

Our research investigates how language bias shapes the information presented across different language editions of Wikipedia. While Wikipedia strives to be a neutral and global knowledge platform, articles often vary significantly in tone, emphasis, and detail depending on the language in which they are written. Using the Wikipedia API, we collected 500-character summaries from various topics—including politics, history, and science—in English, Japanese, and Spanish. By comparing the outputs, we observed that certain details and perspectives were emphasized differently between languages, revealing cultural and linguistic biases embedded within the content. Although our analysis was primarily qualitative, the findings show that Wikipedia is not a direct translation across languages but rather a set of culturally shaped narratives. Understanding these differences is important because language bias can influence global access to reliable information and reinforce unequal knowledge representation online. Our project highlights the need for continued awareness and technical methods to detect and mitigate language bias in multilingual digital platforms.
